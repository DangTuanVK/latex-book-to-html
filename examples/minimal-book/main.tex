\documentclass[a4paper,11pt]{book}

\usepackage{amsmath,amssymb,amsthm}
\usepackage[utf8]{inputenc}
\usepackage{hyperref}

% Theorem environments
\newtheorem{theorem}{Theorem}[chapter]
\newtheorem{lemma}[theorem]{Lemma}
\newtheorem{definition}[theorem]{Definition}
\newtheorem{example}[theorem]{Example}

\title{A Minimal Example Book}
\author{Example Author}
\date{2026}

\begin{document}

\maketitle
\tableofcontents

\part{Foundations}

\chapter{Introduction to Number Theory}

\section{What is Number Theory?}

Number theory is the branch of mathematics that studies the properties of integers.
It is one of the oldest areas of mathematics, dating back to ancient civilizations.

\begin{definition}
A \textbf{prime number} is a natural number $p > 1$ that has no positive divisors
other than $1$ and $p$ itself.
\end{definition}

The first few primes are: $2, 3, 5, 7, 11, 13, 17, 19, 23, 29, \ldots$

\begin{theorem}[Euclid]
There are infinitely many prime numbers.
\end{theorem}

\begin{proof}
Suppose there are finitely many primes $p_1, p_2, \ldots, p_n$.
Consider the number $N = p_1 p_2 \cdots p_n + 1$.
Then $N$ is not divisible by any $p_i$, so either $N$ is prime
or $N$ has a prime factor not in the list. Contradiction.
\end{proof}

\section{The Fundamental Theorem of Arithmetic}

\begin{theorem}[Fundamental Theorem of Arithmetic]
Every integer $n > 1$ can be written uniquely (up to ordering) as a product of primes:
\[
n = p_1^{a_1} p_2^{a_2} \cdots p_k^{a_k}
\]
where $p_1 < p_2 < \cdots < p_k$ are primes and $a_i \geq 1$.
\end{theorem}

\begin{example}
The prime factorization of $360$ is:
\[
360 = 2^3 \cdot 3^2 \cdot 5
\]
\end{example}

An important function in number theory is the \textbf{Euler's totient function} $\varphi(n)$,
which counts the number of integers from $1$ to $n$ that are coprime to $n$:
\[
\varphi(n) = n \prod_{p \mid n} \left(1 - \frac{1}{p}\right)
\]

For a prime $p$, we have $\varphi(p) = p - 1$. More generally, $\varphi(p^k) = p^{k-1}(p-1)$.
\section{Modular Arithmetic}

\begin{definition}
We say that $a$ is \textbf{congruent} to $b$ modulo $n$, written $a \equiv b \pmod{n}$,
if $n$ divides $a - b$.
\end{definition}

\begin{theorem}[Fermat's Little Theorem]
If $p$ is prime and $\gcd(a, p) = 1$, then
\[
a^{p-1} \equiv 1 \pmod{p}
\]
\end{theorem}

\begin{theorem}[Euler's Theorem]
If $\gcd(a, n) = 1$, then
\[
a^{\varphi(n)} \equiv 1 \pmod{n}
\]
\end{theorem}

This generalizes Fermat's Little Theorem since $\varphi(p) = p - 1$ for primes.

See \cite{euler1748} for Euler's original work and \cite{hardy-wright} for a comprehensive treatment.

\chapter{Analytic Number Theory}

\section{The Riemann Zeta Function}

The Riemann zeta function is defined for $\text{Re}(s) > 1$ by the series:
\[
\zeta(s) = \sum_{n=1}^{\infty} \frac{1}{n^s}
\]

One of the most beautiful formulas in mathematics is the \textbf{Euler product}:
\[
\zeta(s) = \prod_{p \text{ prime}} \frac{1}{1 - p^{-s}}, \qquad \text{Re}(s) > 1
\]

This connects the zeta function (an analytic object) with prime numbers (an arithmetic object).

\begin{theorem}[Euler, 1737]
The Euler product formula holds for $\text{Re}(s) > 1$, and in particular,
taking $s = 1$ shows that the series $\sum_p 1/p$ diverges, giving another
proof that there are infinitely many primes.
\end{theorem}

\section{Special Values}

Some famous special values of the zeta function:

\begin{itemize}
\item $\zeta(2) = \dfrac{\pi^2}{6}$ \quad (Basel problem, solved by Euler in 1734)
\item $\zeta(4) = \dfrac{\pi^4}{90}$
\item $\zeta(6) = \dfrac{\pi^6}{945}$
\item $\zeta(2k) = (-1)^{k+1} \dfrac{(2\pi)^{2k} B_{2k}}{2 (2k)!}$ \quad (general formula with Bernoulli numbers $B_{2k}$)
\end{itemize}

\section{The Prime Number Theorem}

\begin{definition}
The \textbf{prime counting function} $\pi(x)$ denotes the number of primes $p \leq x$.
\end{definition}

\begin{theorem}[Prime Number Theorem (Hadamard, de la Vall\'ee-Poussin, 1896)]
\[
\pi(x) \sim \frac{x}{\ln x} \quad \text{as } x \to \infty
\]
More precisely, $\lim_{x \to \infty} \frac{\pi(x) \ln x}{x} = 1$.
\end{theorem}

A table of values comparing $\pi(x)$ with $x / \ln x$:

\begin{center}
\begin{tabular}{|c|c|c|c|}
\hline
$x$ & $\pi(x)$ & $x/\ln x$ & Ratio \\
\hline
$10^3$ & 168 & 145 & 1.16 \\
$10^6$ & 78,498 & 72,382 & 1.08 \\
$10^9$ & 50,847,534 & 48,254,942 & 1.05 \\
$10^{12}$ & 37,607,912,018 & 36,191,206,825 & 1.04 \\
\hline
\end{tabular}
\end{center}

The Prime Number Theorem was first conjectured by Gauss \cite{gauss1801} around 1792
when he was only 15 years old, based on numerical evidence from tables of primes.

See \cite{hardy-wright} for proofs and further discussion.


\begin{thebibliography}{9}
\bibitem{euler1748}
L. Euler, \textit{Introductio in analysin infinitorum}, Lausanne, 1748.

\bibitem{gauss1801}
C.F. Gauss, \textit{Disquisitiones Arithmeticae}, Leipzig, 1801.

\bibitem{hardy-wright}
G.H. Hardy and E.M. Wright, \textit{An Introduction to the Theory of Numbers}, 6th ed., Oxford University Press, 2008.
\end{thebibliography}

\end{document}
