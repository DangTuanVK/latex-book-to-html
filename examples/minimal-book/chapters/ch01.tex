\chapter{Introduction to Number Theory}

\section{What is Number Theory?}

Number theory is the branch of mathematics that studies the properties of integers.
It is one of the oldest areas of mathematics, dating back to ancient civilizations.

\begin{definition}
A \textbf{prime number} is a natural number $p > 1$ that has no positive divisors
other than $1$ and $p$ itself.
\end{definition}

The first few primes are: $2, 3, 5, 7, 11, 13, 17, 19, 23, 29, \ldots$

\begin{theorem}[Euclid]
There are infinitely many prime numbers.
\end{theorem}

\begin{proof}
Suppose there are finitely many primes $p_1, p_2, \ldots, p_n$.
Consider the number $N = p_1 p_2 \cdots p_n + 1$.
Then $N$ is not divisible by any $p_i$, so either $N$ is prime
or $N$ has a prime factor not in the list. Contradiction.
\end{proof}

\section{The Fundamental Theorem of Arithmetic}

\begin{theorem}[Fundamental Theorem of Arithmetic]
Every integer $n > 1$ can be written uniquely (up to ordering) as a product of primes:
\[
n = p_1^{a_1} p_2^{a_2} \cdots p_k^{a_k}
\]
where $p_1 < p_2 < \cdots < p_k$ are primes and $a_i \geq 1$.
\end{theorem}

\begin{example}
The prime factorization of $360$ is:
\[
360 = 2^3 \cdot 3^2 \cdot 5
\]
\end{example}

An important function in number theory is the \textbf{Euler's totient function} $\varphi(n)$,
which counts the number of integers from $1$ to $n$ that are coprime to $n$:
\[
\varphi(n) = n \prod_{p \mid n} \left(1 - \frac{1}{p}\right)
\]

For a prime $p$, we have $\varphi(p) = p - 1$. More generally, $\varphi(p^k) = p^{k-1}(p-1)$.
\section{Modular Arithmetic}

\begin{definition}
We say that $a$ is \textbf{congruent} to $b$ modulo $n$, written $a \equiv b \pmod{n}$,
if $n$ divides $a - b$.
\end{definition}

\begin{theorem}[Fermat's Little Theorem]
If $p$ is prime and $\gcd(a, p) = 1$, then
\[
a^{p-1} \equiv 1 \pmod{p}
\]
\end{theorem}

\begin{theorem}[Euler's Theorem]
If $\gcd(a, n) = 1$, then
\[
a^{\varphi(n)} \equiv 1 \pmod{n}
\]
\end{theorem}

This generalizes Fermat's Little Theorem since $\varphi(p) = p - 1$ for primes.

See \cite{euler1748} for Euler's original work and \cite{hardy-wright} for a comprehensive treatment.
